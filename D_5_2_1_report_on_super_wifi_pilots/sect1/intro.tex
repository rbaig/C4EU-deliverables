\IEEEPARstart{W}{hite Spaces} (WS) is referred to the unoccupied frequencies in the spectrum. TV White Spaces (TVWS) is the term used to identify White Spaces in the TV band (between $471.25$ and $863.25$~MHz). This spectrum availability is the result from the digital switchover of analog TV channels, allowing more TV data to be transmitted over the same $8$~MHz-wide UHF TV channel.

Some of the benefits of attempting transmission over these frequencies are longer ranges and better building penetration than in the unlicensed $2.4$ and $5$~GHz bands used by the IEEE~$802.11$ set of protocols. Furthermore, when compared with higher frequencies, transmissions over TVWS may result in a reduced cost due to its extended coverage area. However, this will depend on the business model and policies implemented.

The Digital Dividend will bring a reduction in the UHF band usage and has been considered by the European Commission as a way of boosting the economic growth of the region~\cite{digitalDividend}. The shift in spectrum will take place in TV channels from $61$ to $69$~($790$~MHz to $862$~MHz). Previous to the Digital Dividend the Commission resolved to take part of the spectrum released by the digital switchover, as shown in their paragraphs 30 to 37, to promote among the European countries special licenses for TVWS frequencies~\cite{spectrumShift}.

% \IEEEPARstart{S}{uper} WiFi is an emerging attempt to make use of the available spectrum in the UHF TV band (between $471.25$ and $863.25$~MHz) for wireless data transmissions. This spectrum availability is the result from the digital switchover of analog TV channels,  allowing more TV data to be transmitted over the same $8$~MHz-wide UHF TV channel. The remaining available spectrum in this band is referred to as TV White Spaces (TVWS).
% 
% Some of the benefits of attempting transmission over these frequencies are longer ranges and better building penetration than in the unlicensed $2.4$ and $5$~GHz bands used by the IEEE~$802.11$ set of protocols. These benefits are followed by strict regulatory and technological challenges regarding the administration of TV White Spaces (TVWS) and cognitive capabilities for incumbent avoidance.

As long as this spectrum was seen by regulation as TV bands, it has never been allowed to be used for anything else than broadcasting television. Before $2009$, FCC, OFCOM and private companies were doing research on this area and conducted real-life tests (see~\cite{chandra2011campus}~\cite{davies2011field}~\cite{dvbtSensing}~and~\cite{digitalDividendCognitiveAccess}). Currently, there are some standards suitable to be implemented for those frequencies that can operate within a UHF channel (e.g.: LTE and Super WiFi). Regulation will be applied to guarantee that no interference is seen on the TV licensed channels and to model the use of the TVWS.

 This report assesses some of Super WiFi's technical challenges, specially those related to incumbent detection and avoidance~\cite{shellhammer2009technical}. As a result, an energy detector on the UHF TV band is implemented using the Universal Software Radio Peripheral (USRP) Ettus USRP-E110~\cite{ettusUSRPE110} as a radio front end for embedded applications (see Figure~\ref{fig:usrp_combined}) and the open source Software Defined Radio (SDR) project GNURadio~\cite{GNURadio}. Furthermore, an instant image of the TV UHF band is derived from the detection (see Figure~\ref{fig:tvChannels}).

Section~\ref{sec:related_work} overviews previous work in this area, followed by Section~\ref{sec:implementation} which details the implementation of a spectrum sensor using the USRP-E110. Results are summarized in Section~\ref{sec:results} while conclusions and future directions are provided in Section~\ref{sec:conclusions}.