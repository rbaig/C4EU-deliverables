\IEEEPARstart{D}{espite} the scepticism of some people about the capacity of community networks (CNs) to incorporate the optical fiber (OF) technology in guifi.net there are many on-going initiatives\footnote{In the guifi.net jargon each of these initiatives is called \emph{a project}} to do so. The fact that some of these projects are already in the stage of being fully operational, bringing of Gbs/s broadband Internet access to places (such as rural areas) where the traditional telcos are currently offering connections of few Mbs/s at most, proves that it is totally feasible to deploy and operate OF infrastructure according to the CNs principals following a bottom-up approach, thus that the aforementioned scepticism is totally unfounded.

The present document reports the presence of OF in guifi.net, paying special attention to the three projects that have been selected as OF pilots~\cite{barcelo2012bub} in the Commons4Europe project and how they have progressed over the first year. The Gurb\footnote{Gurb, population 2.538 hab, density 49,19 hab/km$^{2}$, located in the ''comarca'' of Osona, Catalonia. It is a typical Catalan rural village formed by a few streets and many disseminated farms, some of them rather isolated.} project has been selected as a pilot because it was the first OF project started and the most advanced one. The Vic\footnote{Vic, population 40.900 hab, density 1.336,60 hab/km$^{2}$, the capital of the ''comarca'' of Osona, Catalonia. It is a typical middle size Catalan city of the Catalan rural areas where most of the population lives in the urban area with several industrial parks.} pilot has been selected because it is a case of OF in an urban area. Finally Rub\'{i}\footnote{Rub\'{i}, population 73.979 hab, density 2.290,37 hab/km$^{2}$, located in the ''comarca'' of Vall\`{e}s Occidental, Catalonia. It is a typical middle size Catalan city of the Barcelona surroundings where most of the population lives in the urban with several industrial parks.} has been selected as a case where the project at the moment is blocked.

Several new terms have appeared for this new way of deploying OF such as \emph{Fiber From The Farm (FFTF/FFTx)}\footnote{A play on words (i) referring to the active-\emph{from} vs. passive-\emph{to} role of the end users of the CNs models vs. the traditional telcos models, and (ii) reaffirming the popular origin of the initiative \emph{farm} vs. \emph{home}. } or Bottom-up Broadband (BuB)\footnote{Despite this term does not strictly refer to FO the reference is implicit since many people thinks that FO is the only way to grant the broadband.}. BuB term was introduced in the Digital Agenda for Europe as the result of the guifi.net participation in the Stakeholder Day 2010\footnote{\url{http://ec.europa.eu/digital-agenda/events/cf/dae1009/item-display.cfm}}. All these terms refer to the high degree of the implication of the end user in all the phases of the network deployment and operation.

In FO all connections are end-to-end (Point-To-Point) connections\footnote{Precisely speaking Passive Optical Network (PON) technologies allow Point-To-Multipoint connections. Despite they are widely used, also in guifi.net. For the sake of clarity in this document they are usually treated as a group of PTP links.}. Hence, the active parts concentrate in the edges. While the intercity connections usually form a mesh network, the so called \emph{backbone}, the intracity connections usually form a start, the so called \emph{user loop} or \emph{last mile links}, centred in the nodes of the intercity mesh. OF wires are passive, so all the electronics and logical configurations concentrate in the edges. While the next section focuses on the physical part of the deployments, called \emph{deployments} itself, the following following focuses on the nodes, named \emph{Points-Of-Presence (POPs)}. The \emph{results} section summarises the results already achieved. Finally the \emph{conclusion} reviews the information presented.