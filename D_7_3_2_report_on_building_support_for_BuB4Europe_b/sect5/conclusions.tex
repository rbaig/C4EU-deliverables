According to the initial plans, a formal organisation to support BuB initiatives should have been created. Nonetheless as a result of our meetings with BuB initiatives from all around Europe, but also from North and South America, we concluded that its continuity once the project had finished was not guaranteed mostly because the very limited budgets of such initiatives. Thus, we decided to focus our efforts on putting in place tools that we had identified as needed during our investigations. The results of this change of strategy are a proposal for the systematisation of the resolution of conflicts, a methodology for the balance OPEX and CAPEX among the professionals using the commons infrastructure, the development of the software tools to put this methodology in practice, the translation of the guifi.net license into English, etc.

We are very satisfied with the outcome of WP7 because most of the tools developed have been very welcome by the guifi.net community and some of them have already been adopted by other BuB initiatives. We also consider that the job done is well balanced with the allocated resources. Thus, we consider that WP7 goals have been successfully accomplished.
